\documentclass[greek]{beamer}
%\usepackage{fontspec}
%\newtheorem{definition}{Ορισμός}
\usepackage{amsmath,amsthm}
\usepackage{unicode-math}
\usepackage{xltxtra}
\usetheme{Warsaw}
\usecolortheme{seahorse}
\usepackage{hyperref}
\usepackage{ulem}
\usepackage{xgreek}
\usepackage{pgfpages}
%\setbeameroption{show notes on second screen}
%\setbeameroption{show only notes}

\setsansfont{Times New Roman}
\title{One way to rule them all}
\author[Λόλας]{Κ. Λόλας\inst{1}}
\institute[]
{
  \inst{1}%
  10ο ΓΕΛ ΘΕΣ/ΝΙΚΗΣ (ΠΕ03)
}
\date{Jitsi, Θεσσαλονίκη, Μάρτιος 2021}

\begin{document}
\begin{frame}
  \titlepage
\end{frame}

\section{Η Πρόκληση}
\begin{frame}{Η Πρόκληση}
  Δημιουργία αρχείο pdf με μαθηματική άσκηση:
  \begin{itemize}
    \item Ο Υπολογιστής δεν είναι δικός σας
    \item Έχει internet
    \item Υπάρχει επεξεραστής κειμένου
    \item Μη αξιόπιστο PC
    \item Να μην πάει χαμένος ο κόπος
  \end{itemize}
\end{frame}

\section{Μέθοδος}
\begin{frame}{Browser - Περί ορέξεως}
  \begin{itemize}
    \item Διαλέξτε Browser
    \item Επισκευτείτε τη σελίδα \href{github.com/costasdroid/lectures/PiDay2021/maths.html}{html πρότυπο}.
    \item Αντιγράψτε τον κώδικα
    \item Δεν κλείνουμε τον περιηγητή
  \end{itemize}
\end{frame}

\begin{frame}{Επεξεργαστής Κειμένου - Περί ορέξεως}
  \begin{itemize}
    \item Ανοίξτε τον αγαπημένο σας επεξ. κειμένου
    \item Επικολήστε τον κώδικα από την ιστοσελίδα
    \item Αποθηκεύστε το αρχείο
    \item Ανοίξτε το στον περιηγητή
    \item ΜΑΓΕΙΑ
  \end{itemize}
\end{frame}

\begin{frame}{Δημιουργήστε ελεύθερα}
  \begin{itemize}
    \item Γράψτε ότι μαθηματικά - κείμενο θέλετε
    \item '<br>' για κάθε νέα γραμμή
    \item Ανανεώστε στον περιηγητή για το σωστό αποτέλεσμα
    \item Αποθηκεύστε την δουλειά σας να μην πάει χαμένη
  \end{itemize}
\end{frame}

\begin{frame}{Το αλατοπίπερο}
  Είναι txt αρχείο, οπότε
  \begin{itemize}
    \item Το αντιγράφετε στο word και επιλέγετε μετατροπή $\LaTeX$
    \item Το ανεβάζετε σε δημόσιο αποθετήριο (github...) ώστε να συνεισφέρουν και άλλοι
    \item Το προσθέτετε στους προσωπικούς φακέλους με τα υπόλοιπα $\TeX$ αρχεία
  \end{itemize}
\end{frame}

\begin{frame}[plain,c]
  \begin{center}
    \Huge Σας Ευχαριστούμε...
  \end{center}
\end{frame}

\end{document}
